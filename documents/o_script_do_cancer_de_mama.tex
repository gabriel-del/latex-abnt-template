\chapter{O script do cancer de mama}
\label{chapter:o_script_do_cancer_de_mama}

\section{Validação Cruzada}

Validação cruzada é um método de aprendizado de máquina em que usasse-se uma
porcentagem para teste e outra para treinamanto.

Por exemplo, divide-se o conjunto de dados em $10$ partes e usa-se 9 para fazer o
treinamento e os outros $10\%$ para testar como foi o aprendizado.
Em seguida, usa-se outras $9$ partes do mesmo dado, e testa com a outra parte
que restou, e continua até que tenha-se concluido todas as combinações.


\section{Sintaxe}
O Código está estruturado da seguinte forma:

\begin{lstlisting}[language=Python, caption=Sintaxe Simples]
svm = SVC(kernel='poly',degree=1)
scores = cross_val_score(svm, X, y, cv=10, scoring='accuracy')
function_print = 'SuppotVectorMachine:\t' + str(scores.mean())
print(function_print)
if scores.mean() > best_score:
  best_score = scores.mean()
  best_function=function_print
\end{lstlisting}

Primeiramente, define-se a função que será usada.
Em seguida usa-se a validação cruzada,
exibe-se a média dos resultados.
E por último checa-se se foi a função com o melhor resultado para exibir ao final do script.

Nesta outra função há um parâmetro chamado "max\_depth" que será variado de $1$ a $9$,
para capturar o melhor resultado.

\begin{lstlisting}[language=Python, caption=Sintaxe com Parâmetro]
max_score = 0
for n in range(1,10):
  tree = DecisionTreeClassifier(max_depth=n, random_state=0)
  scores = cross_val_score(tree, X, y, cv=10, scoring='accuracy')
  if  scores.mean() > max_score:
    max_score = scores.mean()
    max_n = n
function_print = 'DecisionTreeClassifier:\t' + str(max_score) + '\t(max_depth=' + str(max_n) + ')'
print(function_print)
if max_score > best_score:
  best_score = max_score
  best_function=function_print
\end{lstlisting}

O código final completo encontra-se no apêndice \ref{app:code}, seção
\ref{lst:code}, página \pageref{lst:code}.

E sua execução resultou nos dados encontrados na tabela \ref{tab:resultado}.

\setlength{\arrayrulewidth}{0.6mm}
\begin{table}[h!]
\centering
\begin{tabular}{ |c|c|c| }
 \hline
 Função                   & Acurácia           & Parâmetro  \\
 \hline
 KneighborsClassifier     & 0.9297619047619048 & n\_neighbors = 8  \\
 DecisionTreeClassifier   & 0.9280701754385964 & max\_depth = 5    \\
 RandomForestClassifier   & 0.9649122807017543 & max\_depth = 80   \\
 SuppotVectorMachine      & 0.9051065162907269 &                   \\
 GaussianNB               & 0.9367794486215537 &                   \\
 MLPClassifier            & 0.8963032581453634 &                   \\
 \hline
 \hline
 \multicolumn{3}{|c|}{ Melhor Função} \\
 \hline
 RandomForestClassifier:  & 0.9649122807017543 & max\_depth = 80   \\
 \hline
\end{tabular}
  \caption{Resultado Final}
\label{tab:resultado}
\end{table}

Observa-se pelos resultados na tabela que com esse código simples, consegue-se uma acurácia de $96\%$
utilizando o classificador "Random Forest" com "Max Depth" de $80$.








