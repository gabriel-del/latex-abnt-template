\chapter{Inteligência Computacional e Saúde}
\label{chapter:inteligencia_computacional_e_saude}

A inteligência Computacional é uma ferramenta de computação que automatiza processos complexos,
como por exemplo a identificação e categorização de imagens. E já tem muito espaço na saúde,
uma área de extrema relevância à sociedade, que requer atenção e agilidade, pois cuida das diversas aplicações neste ramo, estão:

\begin{itemize}
\item Tratamento de doenças: Se refere a capacidade dos computadores de auxiliar no tratamento das patologias.
  Por exemplo o “Watson”, algoritmo desenvolvido pela IBM,
    que informa os vários tratamentos possíveis para cada caso, informando efeitos colaterais e o grau de risco de cada uma.

\item Acurácia no resultado de exames: Existem algoritmos computacionais,
  atualmente, que conseguem ser mais precisos em comparação à alguns exames médicos,
    por exemplo um algoritmo desenvolvido na Alemanha, EUA e China,
    que superou especialistas em retina na identificação de diagnósticos no exame de tomografia óptica

\item Correlação de sintomas: O TensorFlow, por exemplo,
  uma biblioteca desenvolvida pela Google, consegue identificar retinopatia diabética através de semelhança de imagens,
    e consegue fazer associação de sintomas com taxa de acerto semelhante aos especialistas da área.

\item Recuperação de dados: A partir da memória da IC,
  facilita-se a recuperação dos arquivos com os dados e ficha médica dos pacientes, quando houver necessidade.

\item Alertas de emergência: Pode-se criar alertas sobre mudanças no quadro dos pacientes,
  os enviando ao profissional responsável, sendo de extrema importância principalmente em emergências.
\end{itemize}

Para o desenvolvimento da Inteligência Computacional, por meio do Script, que auxiliará na avaliação médica,
foram utilizadas as seguintes ferramentas:

\section{\textbf{Phyton}}

Python é uma linguagem de programação, no ranking das mais utilizadas do mundo, ela foi criada por Guido Van Rossum,
com o objetivo de proporcionar produtividade e legibilidade na realização de projetos computacionais \cite{PYTHON}.

A sua sintaxe é simples, sendo comparada até a um pseudocódigo executável.
Usa a chamada “Identação”, que é um recuo no código, em relação a sua margem, para marcar blocos.
Também é uma linguagem de baixo uso de caracteres especiais e palavras chaves.

Apesar de ser uma linguagem comumente utilizada para IC,
ainda não é simples de encontrar artigos e documentos que instruam para a realização de um programa básico,
a fim de iniciar os primeiros passos na área

Assim, será utilizado no Script de Câncer de mama as seguintes ferramentas básicas e de fácil acesso:

\begin{itemize}
\item Pandas - é uma biblioteca do Python,
  que fornece ferramentas de análise e estruturas de dados de alta performance e de simples utilização \cite{PANDAS}.
\item Scikit-learn - é outra biblioteca do Python que inclui alguns algoritmos de classificação, regressão e agrupamento \cite{SCIKIT}.
\end{itemize}

\section{\textbf{A base de dados}}

Para comprovar a correlação do diagnóstico de câncer a partir dos parâmetros dos exames,
e ensinar a inteligência computacional a tentar descobrir o provável diagnóstico,
foi-se utilizado uma base de dados de autoria da “Breast Cancer Wisconsin” chamada por eles como “Diagnostic Dataset”, presente no site:
(\url{https://www.kaggle.com/uciml/breast-cancer-wisconsin-data})\cite{BREASTCANCER}.

É um conjunto de registros de resultados de exames utilizados para detecção do Câncer de mama,
comumente utilizada para aprendizado e aplicações reais.
Ela foi criada em 1995 por Dr. William H. Wolberg, W. Nick Street e Olvi L. Mangasarian na universidade de Wisconsin.

Possui 569 instâncias, que estão correlacionadas com as pessoas que participaram da coleta.
E 32 tipos de atributos que são os dados coletados de cada análise.
Entre os atributos está o diagnóstico final dado pelo médico informando se o tumor é maligno ou benigno.

Os dados dessa base foram obtidos através de diversos procedimentos de diagnóstico médico.
Dentre eles, há a imagem digitalizada de um aspirado por agulha fina (PAAF)
de um tecido mamário que tem como função determinar os aspectos dos núcleos celulares presentes no tecido.

Essa base é importante para o desenvolvimento do Script do Câncer de mama
pois ela permite analisar os atributos coletados para ensinar a inteligência computacional a tentar descobrir o provável diagnóstico.

Os atributos coletados na base de dados determinam as características do tumor maligno, como:

Raio, textura, perímetro, área, suavidade, compacidade, concavidade e os pontos côncavos presentes;
simetria, além da dimensão fractal.
Para cada um desses parâmetros foram analisados a média, erro padrão e o ``pior'' ou mais largo (média dos 3 maiores valores).


