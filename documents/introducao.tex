\chapter{Introdução}
\label{chapter:introducao}

As neoplasias malignas, também conhecidas como “Câncer” é um conjunto de patologias que,
de acordo com a ORGANIZAÇÃO MUNDIAL DE SAÚDE (OMS, 2018),
é a segunda maior causa de óbitos no mundo,
responsável por 9,6 milhões de mortes, apenas no ano de 2018 \cite{ESTATISTICACANCER}
.

Tal problema a nível global atrai a atenção para a importância de um diagnóstico precoce,
o que aumenta a probabilidade de um tratamento eficaz, pois sabe-se que os pacientes que são diagnosticados em estádios tardios,
podem não responder aos tratamentos curativos.

Para isso, a tecnologia vem sendo utilizada a fim de aprimorar e auxiliar as equipes multidisciplinares de saúde,
principalmente nos diagnósticos médicos.

E é nesta realidade em que uma ferramenta tecnológica vem sendo cada vez mais estudada e aplicada:
A Inteligência Computacional.

Esse ramo da ciência da computação aplicado na área de saúde consegue analisar e definir as variáveis de diversas doenças.
De acordo com Lobo (LOBO, 2017), “A Inteligência Artificial processa esses dados por meio de algoritmos,
que tendem a se aperfeiçoar pelo seu próprio funcionamento,
definindo pelo termo em inglês Self-learning e a propor hipóteses diagnósticas cada vez mais precisas.”
\cite{IAEMSAUDE}

Assim, tendo conhecimento de tais inventos, viu-se a oportunidade da realização de um mecanismo,
que facilitasse a detecção do câncer, auxiliando os profissionais médicos:
Uma Inteligência computacional, idealizada em forma de Script, escalável, escrita na linguagem Python que,
a partir de um banco de dados, aprenda como diferenciar se um tumor é maligno ou benigno.
Está é a motivação da realização desse trabalho que será apresentado,
voltado principalmente para os profissionais que estão começando ou desejam se
aprofundar na área de aprendizagem de máquina em saúde.

Como base, para “ensinar” a inteligência artificial os atributos de um tumor maligno e de tumores benignos,
será utilizada a base de dados da “Breast Cancer Wisconsin” intitulada como “Diagnostic Dataset”,
uma base de dados numérica, o que facilita o aprendizado em comparação a uma base de dados que contenha imagens.
As informações contidas na base,
servirão de referência para, quando for suposto um caso de tumor ao Script, ele consiga calcular a probabilidade,
do mesmo ser de caráter maligno.

Essa análise será feita utilizando-se de seis algoritmos de aprendizagem de máquina, que são: “K-Nearest Neighbors;
Decision Tree; Random Forest; Support Vector Manchine; Naive Bayes e Artificial Neural Network”, eles auxiliarão a inteligência artificial na tomada de decisão do caso proposto.

O artigo está organizado de forma gradativa, começando pelo câncer,
sua definição e importância.
Em seguida sobre a inteligência computacional na área de saúde,
explicações sobre os métodos de análise,
e por último a mostragem do código que analisa a execução de cada função
automáticamente para uma determinada base de dados.

















