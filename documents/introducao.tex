\chapter{Introdução}
\label{chapter:introducao}

As neoplasias malignas, também conhecidas como “Câncer” é um conjunto de patologias qu é a segunda maior causa de óbitos no mundo,
responsável por 9,6 milhões de mortes, apenas no ano de 2018. \cite{ESTATISTICACANCER}

Tal problema a nível global atrai a atenção para a importância de um diagnóstico precoce,
o que aumenta a probabilidade de um tratamento eficaz, pois sabe-se que os pacientes que são diagnosticados em estádios tardios,
podem não responder aos tratamentos curativos.

Para isso, a tecnologia vem sendo utilizada a fim de aprimorar e auxiliar as equipes multidisciplinares de saúde,
principalmente nos diagnósticos médicos.

E é nesta realidade em que uma ferramenta tecnológica vem sendo cada vez mais estudada e aplicada:
A Inteligência Computacional.

Esse ramo da ciência da computação aplicado na área de saúde consegue analisar e definir as variáveis de diversas doenças.
Segundo Lobo: “A Inteligência Artificial processa esses dados por meio de algoritmos,
que tendem a se aperfeiçoar pelo seu próprio funcionamento,
definindo pelo termo em inglês Self-learning e a propor hipóteses diagnósticas cada vez mais precisas.”
\cite{IAEMSAUDE}

Assim, tendo conhecimento de tais inventos, viu-se a oportunidade da realização de um mecanismo,
que facilitasse a detecção do câncer, auxiliando os profissionais médicos:
Uma Inteligência computacional, idealizada em forma de Script, escalável, escrita na linguagem Python que,
a partir de um banco de dados, aprenda como diferenciar se um tumor é maligno ou benigno,
motivando assim a realização desse trabalho, que visa atingir, principalmente, os profissionais de saúde,
iniciantes na área de aprendizagem de máquina.

Como base, para ensinar a inteligência computacional a analisar os atributos que determinam se um tumor é maligno ou benigno,
será utilizada a base de dados da “Breast Cancer Wisconsin” intitulada como “Diagnostic Dataset” \cite{BREASTCANCER},
uma base de dados numérica, o que significa que os atributos já foram extraídos das imagens, portanto não será necessário pré-processamento,
o que facilita o aprendizado.
As informações contidas na base,
servirão de referência para, quando for suposto um caso de tumor ao Script ele consiga calcular a probabilidade,
do mesmo ser de caráter maligno.

Essa análise será feita utilizando-se de seis algoritmos de aprendizagem de máquina, que são: “K Vizinhos Mais Próximos;
Árvore de Decisão; Floresta Aleatória; Máquina de Vetor de Suporte; Naive Bayes e Rede Neural Artificial”,
eles auxiliarão a inteligência computacional na tomada de decisão do caso proposto.

O trabalho estrutura-se de forma a definir as estatísticas do câncer no mundo e seus impactos na saúde mundial,
para retratar a necessidade de tratamentos e diagnósticos eficazes.

Posteriormente, explica-se sobre os avanços das aplicações da IC na área, com seus respectivos métodos de análise;
finalizando com a exibição do código desenvolvido e suas conclusões.
















