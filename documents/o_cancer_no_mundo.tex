\chapter{O Câncer no Mundo}
\label{chapter:o_cancer_no_mundo}

Denominado, também, como neoplasia ou tumor maligno, o câncer é um conjunto de mais de 100 doenças, em que há um crescimento exacerbado de células em um determinado tecido. Essas células, por sua vez, podem se espalhar a tal ponto, que chegam a atingir outros órgãos e tecidos, fenômeno denominado de “metástase”, demonstrando o poder agressivo dessa classe de patologias. (INCA, 2019)\cite{OQUEECANCER}
.

Segundo dados divulgados pela Internacional Association of câncer Registries $(IACR, 2018)$ era estimado para o ano de $2018$,
uma incidência de $18,1$ milhões de novos casos de câncer no mundo $(IACR, 2018 apud ALMEIDA, 2018)$.
Também foi datado, no mesmo ano, $9,6$ milhões de óbitos \cite{MOC}
, como mostrado na figura:

\begin{figure}[!htb]
\begin{center}
\caption{Incidência mundial de câncer}
\includegraphics[width=12cm]{cancerdata}
\end{center}
\legend{Fonte: https://www.iarc.fr/wp-content/uploads/2018/09/Globocan\_01.jpg}
\cite{GLOBOCAN}
\end{figure}



\section{\textbf{Etapas do diagnóstico de câncer}}

O diagnóstico tem como definição ser o processo de identificação a partir de exames médicos, sintomas e sinais além das amostras do tecido atingido. (BRASIL, 2017) \cite{ATLAS}

Assim, até que o Patologista, especialista médico responsável pelo laudo anatomopatológico, faça sua resolução, há alguns processos que auxiliam na análise do problema. Dentre os exames mais solicitados, está a retirada de um fragmento do tecido suspeito, para que possa ser feita uma análise em laboratório no microscópio afim de verificar aspecto dos núcleos e a morfologia das células presentes naquela amostra; exames de sangue, da medula óssea; radiografia; ultrassonografia, ressonância magnética; tomografia computadorizada por emissão de pósitrons (PET-TC), são exemplos de exames convencionais. (INSTITUTO VENCER O CÂNCER, 2017). \cite{VENCER}

É importante destacar que a solicitação desses procedimentos irá variar de acordo com a suspeita do tumor, além da necessidade de uma avaliação mais abrangente, pois há situações em que os exames convencionais não expõem a informação necessária para um diagnóstico definitivo. E em alguns casos é preciso a realização de exames Imuno-histoquímicos (IHQ), com marcações especiais.

Todo esse processo é de extrema importância, pois, apenas a partir da análise desses laudos, pelo médico responsável, que ele poderá, confirmar ou descartar a hipótese da neoplasia, e em caso de confirmação, iniciar um plano de tratamento específico para o problema.


\section{\textbf{Importância do diagnóstico precoce}}

O processo de identificação, exame e avaliação do câncer requer vários fatores, e, portanto, requer tempo.

Porém, profissionais e órgãos da área de saúde alertam para a importância de um diagnóstico precoce, pois esse, pode elevar as chances de um prognóstico favorável ao paciente.

O Ministério da Saúde do Brasil relata a importância de um diagnóstico em estágios iniciais da seguinte maneira:

“O objetivo do diagnóstico precoce é identificar pessoas com sinais e sintomas iniciais da doença, primando pela qualidade e pela garantia da assistência em todas as etapas da linha de cuidado da doença. O diagnóstico precoce, portanto, é uma estratégia que possibilita terapias mais simples e efetivas, ao contribuir para a redução do estágio de apresentação do câncer. Assim, é importante que a população em geral e os profissionais de saúde reconheçam os sinais de alerta dos cânceres mais comuns, passíveis de melhor prognóstico se descobertos no início. A maioria dos cânceres é passível de diagnóstico precoce mediante avaliação e encaminhamento após os primeiros sinais e sintomas.” (BRASIL, 2018)\cite{DIAGNOSTICO}

