\chapter{Conclusão}
\label{chapter:conclusao}

O desenvolvimento do presente trabalho possibilitou uma análise através do realização de um script simples de Inteligência computacional
na área da saúde, que visa permitir um apoio na tomada de decisão médica através de atributos morfológicos de tumores de mama,
contribuindo para uma reflexão da necessidade de criação de atalhos que permitam um rápido diagnóstico,
diante das estatísticas alarmantes de mortalidade de câncer no mundo,

Com a execução da versão final do código, foi possível atingir o objetivo principal,
onde, a partir das seis funções aplicadas no código, que permitiram o aprendizado da Inteligência artificial,
por meio da base de dados das neoplasias, o script coletava as características inseridas para análise e lançava, em porcentagem,
a probabilidade de malignidade,
chegando ao resultado de $96\%$ de acurácia
utilizando o classificador "Random Forest" com "Max Depth" de $80$.


Apesar de existir outros fatores relevantes além da acurácia,
percebe-se que a intenção de realizar uma ferramenta com um valor probabilístico razoável para servir de apoio na detecção de câncer foi alcançada,
considerando que o diagnóstico final é, necessariamente de responsabilidade médica,
pois sabe-se que a análise tumoral envolve outros aspectos, não tangíveis no script.

Tendo em vista a importância do tema abordado, é de extrema relevância a continuidade do desenvolvimento de projetos tecnológicos,
que colaborem para uma rápida avaliação dos casos.

Nesse sentido, a utilização de recursos tecnológicos simples, como o script do câncer de mama permitem um bom auxílio na detecção de câncer,
contribuindo para os profissionais médicos na busca pelo fornecimento de um prognóstico favorável,
diminuindo assim, os dados alarmantes de mortes por neoplasias à nível mundial.
