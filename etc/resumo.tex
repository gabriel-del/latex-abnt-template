Tendo em vista que a necessidade de um diagnóstico precoce de câncer é primordial para um tratamento efetivo,
e consequentemente a cura da neoplasia,
pesquisa-se sobre o desenvolvimento de uma Inteligência Computacional escrita na linguagem Phyton,
com o intuito de criar um mecanismo de apoio ao diagnóstico médico.
Para tanto, é necessário o desenvolvimento de um script com funções de aprendizado de máquina,
aplicação do banco de dados “Diagnostic Dataset” da Breast Cancer Wisconsi (1992).
Além da utilização de algoritmos computacionais,
que possibilitarão a análise por meio da ferramenta, para que assim,
ela informe se determinado conjunto de atributos tumorais tem alta probabilidade de ser maligno.
Diante disso, após sua execução, verifica-se que o script apresenta, uma acurácia de 96\%,
o que impõe a constatação de que é possível o desenvolvimento de um script que auxilie o diagnóstico médico com uma taxa de exatidão relevante.
\vspace{\onelineskip}

\noindent

\textbf{Palavras-chave:} Inteligência computacional. Apoio ao diagnóstico. Script. Phyton. Câncer de mama.
