\addtocounter{page}{+1}
\begin{center}

Nome do Aluno

\vspace{1cm}

\textbf{Título do Trabalho}

\end{center}

\vspace{.4cm}

\begin{flushright}
\parbox{8cm}{
\singlespacing{Dissertação apresentada, como requisito\linebreak parcial para obtenção do título de Mestre em Ciências, ao Programa de Pós-Graduação em Engenharia Mecânica, da Universidade do Estado do Rio de Janeiro. Área de\linebreak concentração: Fenômenos de Transporte}.
}
\end{flushright}

\vspace{.6cm}


% insira abaixo a data de sua defesa
% Caso não tenha defendido ainda, deixe em branco

\noindent Aprovado em: 29 de Maio de 2012

\noindent Banca Examinadora:


%
%
% Os professores da UERJ DEVEM ser citados primeiro, independente de quem seja o orientador.
%
%



\vspace{.7cm}

\begin{flushright}
\parbox{12cm}{

\singlespacing

\hrulefill \\

\vspace{-.4cm}
Prof. Dr. Nome do Professor 1 (Orientador)
\newline
Instituto de Matemática e Estatística da UERJ
\vspace{.7cm}

\hrulefill \\

\vspace{-.4cm}
Prof. Dr. Nome do Professor 2
\newline
Faculdade de Engenharia da UERJ
\vspace{.7cm}

\hrulefill \\

\vspace{-.4cm}
Prof. Dr. Nome do Professor 3
\newline
Universidade Federal do Rio de Janeiro - UFRJ - COPPE
\vspace{.7cm}

\hrulefill \\

\vspace{-.4cm}
Prof. Dr. Nome do Professor 4
\newline
Instituto de Geociências da UFF
\vspace{.7cm}

\hrulefill \\

\vspace{-.4cm}
Prof. Dr. Nome do Professor 5
\newline
Universidade Federal do Rio de Janeiro - UFRJ - COPPE
\vspace{.7cm}

}
\end{flushright}
\vfill

\begin{center}
Rio de Janeiro\linebreak 2012
\end{center}