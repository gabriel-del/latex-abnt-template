In view of the necessity for early diagnosis of cancer is primordial for effective treatment and,
consequently,
the cure of cancer,
research on the development of a Computational Intelligence written in Phyton language,
in order to create a support mechanism to medical diagnosis.
For this,
it’s necessary to develop a script with machine learning functions,
application of Breast Cancer Wisconsi's “Diagnostic Dataset” (1992).
In addition to the use of computational algorithms,
which will enable analysis through the tool,
so that it can inform if a certain set of tumor attributes is likely to be malignant.
Given this, after its execution, it appears that the script has an accuracy of 96\%,
what imposes the realization that it is possible to develop a script that helps the medical diagnosis with a rate of relevant accuracy.
\vspace{\onelineskip}

\noindent
\textbf{Key-words}: Computational Intelligence. Diagnostic Support. Script. Phyton. Breast Cancer.
